% to count words in the abstract: you use the texcount command in the terminal!! super useful

% --------------------- document type and size of the margins: ---------------------
\documentclass[10pt,a4paper,notitlepage,twoside]{article}  
\usepackage[top=2.5cm,left=2cm,right=2cm,bottom=2.5cm]{geometry} % redefines page margins

% --------------------- to allow double spaced text (much better looking for an abstract) ---------------------
\usepackage{setspace}

% ============= START OF DOCUMENT HERE =============
\begin{document}

\begin{center}
\large{\textbf{Plant water-stress parameterization determines the strength of land-atmosphere coupling}}

\ 

Marie Combe$^{a}$, Jordi Vil{\`a}-Guerau de Arellano$^{a}$, Huug G. Ouwersloot$^{b}$, Wouter Peters$^{a,c}$
\normalsize

\

\textit{$^{(a)}$ Meteorology and Air Quality Section, Wageningen University, Wageningen, the Netherlands}

\

\textit{$^{(b)}$ Atmospheric Chemistry Department, Max Planck Institute for Chemistry, Mainz, Germany}

\

\textit{$^{(c)}$ Centre for Isotope Research, University of Groningen, Groningen, the Netherlands}

\

\end{center}

\

\doublespacing % this enables double spacing for the following text

Land-surface models that are currently used in numerical weather predictions models and earth system models all assume various plant water-stress parameterizations. We investigate the impact of this variety of parametrizations on the performance of atmospheric models. For this, we use a conceptual framework where a convective atmospheric boundary-layer (ABL) model is coupled to a daytime model for the land surface fluxes of carbon, water, and energy. We first validate our coupled model for a set of surface and upper-atmospheric diurnal observations over a grown maize field in the Netherlands. We then perform a sensitivity analysis of this coupled land-atmosphere system by varying the modeled plant water-stress response from a very insensitive to a sensitive response during dry soil conditions. We first propose and verify a feedback diagram that ties plant water-stress response and large-scale atmospheric conditions to the diurnal cycles of ABL CO$_2$, humidity and temperature. Based on our undertanstanding of the diurnal coupled system, we then explore the impact of the assumed water-stress reponse for the development of a dry spell on a synoptic time scale. We find that during a progressive 3-week soil drying caused by evapotranspiration, an insensitive plant will dampen atmospheric heating because the vegetation continues to transpire while soil moisture is available. In contrast, the sensitive plant reduces its transpiration to prevent soil moisture depletion. But when absolute soil moisture comes close to wilting point, the insensitive plant will suddenly close its stomata causing a switch to a land-atmosphere coupling regime dominated by sensible heat exchange. We find that in both cases, our modeled progressive soil moisture depletion contributes to further atmospheric warming up to 6~K, reduced photosynthesis up to 89\,\%, and CO$_2$ enrichment up to~30~ppm, but the full impact is strongly delayed for the insensitive plant. Finally, we demonstrate that the assumed plant water-stress parametrization can strongly shift the model sensitivity to other environmental conditions that are coherent with a soil drying during droughts (e.g. larger atmospheric subsidence, decreased cloud cover, etc). Our findings thus indicate that the ability of coupled weather or climate models to represent drought events is very much tied to the underlying plant water-stress parametrization.

\end{document}